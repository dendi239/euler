\documentclass[11pt,a4paper]{article}

\usepackage[T2A]{fontenc}
\usepackage[utf8]{inputenc}
\usepackage[russian]{babel}

\usepackage{amsmath}
\usepackage{blindtext}
\usepackage{hyperref}
\usepackage{numprint}

\usepackage{listings}
\usepackage{xcolor}

\definecolor{codegreen}{rgb}{0,0.6,0}
\definecolor{codegray}{rgb}{0.5,0.5,0.5}
\definecolor{codepurple}{rgb}{0.58,0,0.82}
\definecolor{backcolour}{rgb}{0.95,0.95,0.92}

\lstdefinestyle{mystyle}{
    backgroundcolor=\color{backcolour},   
    commentstyle=\color{codegreen},
    keywordstyle=\color{magenta},
    numberstyle=\tiny\color{codegray},
    stringstyle=\color{codepurple},
    basicstyle=\ttfamily\footnotesize,
    breakatwhitespace=false,         
    breaklines=true,                 
    captionpos=b,                    
    keepspaces=true,                 
    numbers=left,                    
    numbersep=5pt,                  
    showspaces=false,                
    showstringspaces=false,
    showtabs=false,                  
    tabsize=2
}

\lstset{style=mystyle}

\newcommand{\pycode}[1]{\lstinputlisting[language=Python]{#1}}

\newenvironment{task}[2]
{\section{#1}
\newcommand{\printcode}{\pycode{#2}}
}
{\printcode
}

\title{Решения первых десяти задач из \href{https://projecteuler.net/index.php?section=problems}{Project Euler}}
\author{Денис Смирнов}
\date{}

\begin{document}

\maketitle
\tableofcontents
\clearpage

\begin{task}{Кратные трём и пяти}{001-multiples-3-and-5.py}
    Если мы выпишем все натуральные числа меньше 10, которые делятся на 3 или 5, получим 3, 5, 6 и 9.
    Сумма этих чисел равна 23.

    Найдите сумму всех чисел, делящихся на 3 или 5 меньших 1000.
\end{task}

\begin{task}{Чётные числа фиббоначчи}{002-even-fibbonacci-numbers.py}
    Каждый новый элемент в последовательности Фиббоначчи получается сложением двух предыдущих элементов.
    Начиная с 1 и 2, первые 10 элементов будут равны:
    \begin{equation*}
        1, 2, 3, 5, 8, 13, 21, 34, 55, 89, \dots
    \end{equation*}

    Рассматривая числа Фиббоначчи не превосходящие 4 миллиона, найдите сумму чётных элементов.
\end{task}

\begin{task}{Наибольший простой делитель}{003-largest-prime-factor.py}
    Простые делители числа 13195 это 5, 7, 13 и 29.
        
    Какой наибольший простой делитель числа 600851475143?
\end{task}

\begin{task}{Наибольшее произведение-палиндром}{004-largest-palindrome-product.py}
    Числа-палиндромы читаются одинаково слева-направо и справа-налево.
    Наибольший палиндром полученный как произведение двух двузначных чисел это $9009 = 91 \times 99$.

    Найдите наибольший палиндром полученный как произведение двух трёхзначных чисел.
\end{task}

\begin{task}{Наименьшее общее кратное}{005-smallest-multiple.py}
    2520 --- это наименьшее число которое делится без остатка на все числа от 1 до 10.

    Найдите наименьше общее кратное всех чисел от 1 до 20.
\end{task}

\begin{task}{Сумма квадратов разностей}{006-sum-square-difference.py}
    Сумма квадратов первых 10 натуральных чисел это:
    \begin{equation*}
        1^2 + 2^2 + \dots + 10^2 = 385
    \end{equation*}
    Квадрат суммы первых 10 натуральных чисел это:
    \begin{equation*}
        \left(1 + 2 + \dots + 10\right)^2 = 3025
    \end{equation*}
    Отсюда, разница между суммой квадратов и квадратом суммы первых 10 натуральных чисел это:
    \begin{equation*}
        3025 - 385 = 2640
    \end{equation*}

    Найдите разницу между суммой квадратов и квадратом суммы первых 100 натуральных чисел.
\end{task}

\begin{task}{10001-ое простое}{007-10001st-prime.py}
    Выписывая первые 6 простых чисел: 2, 3, 5, 7, 11 и 13, мы видим что 6ое простое число это 13.

    Найдите \numprint{10001}ое простое число.
\end{task}

\begin{task}{Наибольшее произведение подотрезка}{008-largest-product-in-a-series.py}
    Четыре последовательных цифры в 1000-значном числе с наибольшим произведением это: $9\times 9 \times 8 \times 9 = 5832$.
    Само число, однако, указано в переменной \texttt{N} в ниже в коде.

    Найдите тринадцать последовательных цфир с наибольшим произведением.
    Какое значение этого произведения?
\end{task}

\begin{task}{Особая тройка Пифагора}{009-special-pythagorean-triplet.py}
    Тройка Пифагора это множество из трёх натуральных чисел, $a < b < c$, для которых:
    \begin{equation*}
        a^2+b^2 = c^2
    \end{equation*}
    Например, $3^2+4^2=9+16=25=5^2$.

    Существует ровно одна Пифагорова тройка для которой $a+b+c=1000$.
    Найдите произведение $abc$.
\end{task}

\begin{task}{Сумма простых}{010-summation-of-primes.py}
    Сумма простых меньших 10 это: $2+3+5+7=17$.

    Найдите сумму простых до двух миллионов.
\end{task}

\end{document}
